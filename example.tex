\documentclass{article}

\usepackage{amsmath} % For mathematical symbols and expressions
\usepackage{graphicx} % For including images

\title{An Example LaTeX Document}
\author{John Doe}
\date{\today}

\begin{document}

\maketitle

\begin{abstract}
This is a brief abstract of the document. It provides a summary of the content.
\end{abstract}

\section{Introduction}
This is the introduction section. Here, you can introduce the topic of your document.

\section{Main Content}

\subsection{Lists}
You can create different types of lists in LaTeX:

\begin{itemize}
    \item Item 1
    \item Item 2
    \item Item 3
\end{itemize}

\begin{enumerate}
    \item First item
    \item Second item
    \item Third item
\end{enumerate}

\subsection{Mathematical Expressions}
LaTeX is great for typesetting mathematical expressions. Here are a few examples:

Inline math: \( E = mc^2 \)

Displayed equations:
\begin{equation}
    a^2 + b^2 = c^2
\end{equation}

\begin{align}
    f(x) &= x^2 + 2x + 1 \\
    g(x) &= \frac{1}{x}
\end{align}

\subsection{Including Images}
You can also include images in your document:

\begin{figure}[h!]
    \centering
    \includegraphics[width=0.5\textwidth]{example-image}
    \caption{An example image.}
    \label{fig:example}
\end{figure}

\section{Conclusion}
This is the conclusion section. Summarize your findings or restate the main points of your document.

\end{document}
